\documentclass[12pt]{article}
\usepackage[sorting = none, style = ieee, doi = false]{biblatex}
\usepackage{csquotes}
\usepackage{tocloft}
\usepackage{fancyhdr}
\usepackage{caption}
\usepackage{amsmath}
\usepackage{listings}
\usepackage{xcolor}
\usepackage{enumitem}
\usepackage{soul}
\usepackage{lipsum}
\usepackage{graphicx}
\usepackage{pythonhighlight}
\usepackage{multirow}
\usepackage{algorithm}
\usepackage{bm}
\usepackage{algpseudocode}
\usepackage{lscape}
\usepackage{multicol}
\usepackage{enumitem}
% klikabilan TOC i reference
\usepackage[croatian]{babel}
\PassOptionsToPackage{hyphens}{url}\usepackage{hyperref}
\usepackage{xurl}

%--------postavke dokumenta--------------
\usepackage
[
a4paper,
left = 25mm,
right = 20mm,
top = 25mm,
bottom = 25mm
]
{geometry}

\newcommand{\signature}[2][9em]{%
  \begin{tabular}[t]{ p{#1} p{#1} }
    \strut\raggedleft
    \raisebox{-.5ex}[0pt][0pt]{\bfseries #2} & \\
    \cline{2-2}
    & \centering\scriptsize\itshape \small{(Luka Kaučić)}
  \end{tabular}
}



\renewcommand{\baselinestretch}{1.5}
\addbibresource{library.bib}
\numberwithin{equation}{section}
\renewcommand{\theequation}{\thesection-\arabic{equation}}
\newcommand{\brix}{$^{\circ}$\,Brix}

\pagestyle{fancy}
%\renewcommand{\headrulewidth}{0pt}
\fancyhead{}
\fancyfoot{}
\fancyfoot[R]{\thepage}
%---------------------------------------

%following two lines add break in bibpliography list for long url-s and similar
\setcounter{biburllcpenalty}{7000}
\setcounter{biburlucpenalty}{8000}



\counterwithin{figure}{section}
\setlength{\belowcaptionskip}{-15pt}
\counterwithin{table}{section}
\captionsetup[table]{skip=3pt}

\renewcommand\thefigure{\thesection.\arabic{figure}.}
\renewcommand\thetable{\thesection.\arabic{table}.}
\captionsetup[figure]{labelsep=space}
\captionsetup[table]{labelsep=space}

\begin{document}


\setlength{\headheight}{15pt}
\pagestyle{fancy}
\fancyhf{}
\fancyhead[L]{\leftmark}
\fancyhead[R]{Luka Kaučić}
\fancyfoot[R]{\thepage}

% Define custom colors
\definecolor{codegreen}{rgb}{0.2,0.6,0.2}
\definecolor{codegray}{rgb}{0.5,0.5,0.5}
\definecolor{backcolor}{rgb}{0.95,0.95,0.95}

% Define JavaScript language settings for listings package
\lstdefinelanguage{JavaScript}{
    keywords={const, let, var, function, new, if, else, for, while, do, switch, case, break, continue, return},
    keywordstyle=\color{blue}\bfseries,
    ndkeywords={class, export, boolean, throw, implements, import, this},
    ndkeywordstyle=\color{darkgray}\bfseries,
    identifierstyle=\color{black},
    sensitive=false,
    comment=[l]{//},
    morecomment=[s]{/*}{*/},
    commentstyle=\color{purple}\ttfamily,
    stringstyle=\color{olive}\ttfamily,
    morestring=[b]',
    morestring=[b]"
}

% Set default settings for code listings
\lstset{
    language=JavaScript,
    basicstyle=\ttfamily\small,
    breaklines=true,
    keywordstyle=\color{blue}\bfseries,
    commentstyle=\color{purple}\itshape,
    stringstyle=\color{olive},
    numbers=left,
    numberstyle=\tiny\color{gray},
    frame=single,
    showspaces=false,
    showstringspaces=false,
    tabsize=2,
    captionpos=b,
    xleftmargin=16pt,
    framexleftmargin=16pt,
    framexrightmargin=0pt,
    framexbottommargin=4pt,
    framextopmargin=4pt,
}

\renewcommand{\lstlistingname}{Izlistanje}
\setlength\cftparskip{8pt} %prored u TOC
%\selectfont

\begin{titlepage}
\pagenumbering{gobble}
    \begin{center}
        \large\textbf{SVEUČILIŠTE JOSIPA JURJA STROSSMAYERA U OSIJEKU}\\[16pt]
         \large\textbf{FAKULTET ELEKTROTEHNIKE, RAČUNARSTVA I INFORMACIJSKIH TEHNOLOGIJA OSIJEK}\\[1in]
         \large\textbf{Sveučilišni diplomski studij računarstva}\\
         \large\textbf{Vizualizacija podataka}
         \vspace*{3cm}

        \sc\LARGE\textbf{Vizualizacija statistike Nobelovih nagrada}
        
        \vspace*{1cm}
        \large\textbf{Projektni zadatak}\\
        \vspace*{2.5cm}
        \Large\textbf{Luka Kaučić}\\
        \vspace*{3cm}
        \normalsize\textbf {Osijek, 2023.}

    \end{center}
\end{titlepage}
\newpage
\addtocontents{toc}{\protect\thispagestyle{empty}}
\tableofcontents
\thispagestyle{empty}
\newpage
\newpage
\pagenumbering{arabic}
\section{DEFINIRANJE PROJEKTNOG ZADATKA}
Ovaj projektni zadatak napravljen je u sklopu kolegija Vizualizacija podataka na Sveučilišnom diplomskom studiju računarstva. U nastavku poglavlja dat će se pregled projektnog zadatka, odabranog podatkovnog skupa, kao i postupaka predobrade istog. Na kraju će se po dostupnim podacima i sličnim projektima dostupnima na internetu dati prijedlog vrsta prikaza koje će se u ovom radu koristiti. 
\subsection{Projektni zadatak}
Za temu projektnog zadatka odabrana je vizualizacija statistike Nobelovih nagrada. Potrebno je dati pregled osvojenih Nobelovih nagrada po državama svijeta, na način da se jasno može razlikovati količina osvojenih Nobelovih nagrada po zemljama. Nadalje, potrebno je dati i prikaz statistike po spolovima za odabranu zemlju i kategoriju. Izvorni kod za projekt nalazi se na sljedećoj poveznici: \url{https://github.com/lkaucic/nobelPrizesD3.git}
\subsection{Podaci i predobrada}
Podatkovni skup preuzet je s \url{https://www.kaggle.com/datasets/rishidamarla/nobel-prize-winners-19002020?resource=download&select=nobel_prize_by_winner.csv} u formatu CSV datoteke, a sadrži sljedeće stupce:
\begin{multicols}{3}
\begin{enumerate}
    \item Id
    \item Firstname
    \item Surname
    \item Born
    \item Died
    \item BornCountry
    \item BornCountryCode
    \item BornCity
    \item DiedCountry
    \item DiedCountryCode
    \item Gender
    \item Year
    \item Category
    \item OverallMotivation
    \item Share
    \item Motivation
    \item Name
    \item City
    \item Country
\end{enumerate}
\end{multicols}
\newpage
Prilikom predobrade podataka uviđeno je sljedeće:
\begin{itemize}
\item dobitnici su muškarci/žene i organizacije; radi jednostavnosti i smislenosti vizualizacije, organizacije se neće razmatrati

\item na pojedinim unosima nisu navedeni godina, područje za koje je osoba dobila Nobelovu nagradu te većina ostalih podataka; s obzirom na to, takvi unosi neće se razmatrati

\item većina vrijednosti stupca „overallMotivation“ nedostaju pa se taj stupac u cjelosti neće razmatrati

\item cilj projekta je prikazati podatke u kontekstu zemalja, drugim riječima relevantni podaci vezani su uz zemlju, spol, godinu i područje dobivene nagrade; s obzirom na to, podaci pripadajućih institucija te  određenih osobnih podataka neće se razmatrati 

\item Korišten standard za notaciju država je alpha-2 (dva slova, npr. GB) dok je u geojson datoteci korišten alpha-3 standard (tri slova, npr. GBR); prema tome, bilo je potrebno prepraviti standard u korištenom podatkovnom skupu (zato što je isto jednostavnije od promjene standarda u geojson datoteci)
\end{itemize}
Nakon predobrade podaci su svedeni na sljedeće stupce:
\begin{multicols}{2}
    \begin{enumerate}
        \item Firstname
        \item Surname
        \item BornCountry
        \item BornCountryCode
        \item Gender
        \item Year
        \item Category
        \item Motivation
    \end{enumerate}
\end{multicols}
\subsection{Relevantne vrste prikaza za korištene podatke}
S obzirom da za svakog dobitnika postoje podaci po zemljama i područjima, moguće je prikazati podatke na karti svijeta. Definiranjem skale boje na istoj moguće je prikazati zemlje s manje dobivenih nagrada (slabiji intenzitet boje) i s više (jači intenzitet boje). Podaci o odnosu osvojenih nagrada po kategorijama ili spolu može se prikazati tortnim dijagramom.
\newpage
\section{DIZAJN VIZUALIZACIJE PODATAKA}
U ovom poglavlju istražuju se ključni aspekti dizajna vizualizacije statistike Nobelovih nagrada u D3 JavaScript okruženju. Fokus je usmjeren na postavljanje ciljeva vizualizacije, razvoj skice, pregled postojećih rješenja i odabir boja za optimalnu prezentaciju podataka. Cilj je stvoriti informacijski bogatu i vizualno privlačnu vizualizaciju. Prvo će se baviti pitanjima na koja želi pružiti odgovor putem vizualizacije, kako bi se usmjerila pažnja na relevantne aspekte podataka i korisničke potrebe. Zatim će se istražiti proces razvoja skice vizualizacije, koji omogućuje testiranje ideja i osiguravanje jasne komunikacije informacija o Nobelovim nagradama. Pregled postojećih rješenja pružit će inspiraciju i uvid u najbolje prakse koje se mogu primijeniti. Također će se istražiti važnost odabira boja za poboljšanje čitljivosti, naglašavanje ključnih podataka i stvaranje ugodnog korisničkog iskustva.
\subsection{Pitanja na koja vizualizacija daje odgovor}
U ovom potpoglavlju istražuju se ključna pitanja na koja se odgovor pruža putem vizualizacije statistike Nobelovih nagrada. Proučavaju se raspodjela dobitnika Nobelove nagrade prema različitim kriterijima te se analiziraju promjene kroz godine i među zemljama. Ovdje su iznesena pitanja koja se istražuju:

\begin{itemize}
\item Koja zemlja ima više, a koja ima manje dobitnika Nobelove nagrade po području?
\item Kakav je omjer muškaraca i žena koji su dobitnici Nobelove nagrade?
\item Kakva je statistika dobivenih Nobelovih nagrada po području za odabranu zemlju?
\item Kako se raspodjela dobitnika Nobelovih nagrada mijenjala kroz godine po zemljama?
\item Kako se raspodjela dobitnika Nobelovih nagrada mijenjala kroz godine po spolu?
\end{itemize}
Ova pitanja pružaju uvid u različite aspekte statistike Nobelovih nagrada i omogućuju da se razumije raspodjela nagrada na temelju različitih kriterija. Putem vizualizacije podataka, dobiva se cjelovita slika i bolje razumijevanje važnih trendova i karakteristika povezanih s Nobelovim nagradama.
\subsection{Skica vizualizacije podataka}
Glavni dio vizualizacije je interaktivna karta svijeta. Korisnik može odabrati za koju će se kategoriju, tj. Nobelovu nagradu prikazati podaci na karti. Prelaskom mišem preko određene zemlje prikazuje se \textit{pop-up} s podatkom o imenu zemlje te broju osvojenih nagrada za odabranu kategoriju. Klikom na isu zemlju pored karte se prikazuje tortni dijagram koji prikazuje raspodjelu osvojenih nagrada po spolu. Uz navedeno radi estetike implementirat će se i tranzicije/animacije.
\subsection{Pregled postojećih rješenja}
U ovom poglavlju će se pružiti pregled različitih izvora i resursa koji su se bavili vizualizacijom statistike Nobelovih nagrada. Različiti pristupi i perspektive u vizualizaciji podataka o Nobelovim nagradama bit će obuhvaćeni. Specifičnosti će biti obrađene u svakom izvoru. Uključeni su izvori koji sadrže programski kod kako bi omogućili korisnicima da sami implementiraju vizualizacije i istražuju podatke. Na primjer, izvor LAV30 \cite{LAV30} uključuje programski kod koji može biti koristan za implementaciju vizualizacija Nobelovih nagrada u određenom programskom jeziku ili okviru. Primjeri i tehnike vizualizacije podataka o Nobelovim nagradama koristeći Python pruženi su u izvoru abigailchen na RPubs \cite{abigailchen}. Analiza određenih aspekata statistike Nobelovih nagrada obrađena je u nekim izvorima. Na primjer, raspodjela dobitnika Nobelove nagrade prema spolu i kategorijama istražena je u izvoru stats.areppim.com \cite{stats.areppim}. Slične informacije pružene su u izvoru statista.com \cite{statista} gdje se pružaju vizualni prikazi i statistička analiza zastupljenosti spolova među laureatima Nobelove nagrade. Interaktivne vizualizacije i alati za istraživanje podataka vezanih uz Nobelovu nagradu dostupni su u nekim izvorima. Na primjer, interaktivne vizualizacije koje se fokusiraju na strukturu i povijest svemira pružene su u izvoru nobelprize.org \cite{nobelprize.org}, dok izvor acanimal na GitHubu \cite{acanimal-github} pruža interaktivne vizualizacije i alate za istraživanje podataka o Nobelovim laureatima koristeći d3.js, JavaScript knjižnicu za vizualizaciju podataka. Također, neki izvori se bave specifičnim pitanjima ili temama vezanim uz Nobelovu nagradu. Na primjer, izvor ARomoH na GitHubu \cite{ARomoH} pruža vizualizacije podataka o Nobelovim laureatima, dok se izvor Jagadish Katam na LinkedInu \cite{JagadishKatam} bavi vizualizacijom statistike dobitnika Nobelove nagrade prema dobi koristeći R programski jezik. Također, izvor DataCamp \cite{DataCamp} pruža vodiče za analizu dobitnika Nobelove nagrade koristeći R. Pregled izvora također uključuje analizu povijesti Nobelove nagrade i njezinih dobitnika. Na primjer, izvor acuriousanimal.com \cite{acuriousanimal} pruža vizualno istraživanje povijesti Nobelove nagrade. Važno je napomenuti da ovi izvori pružaju širok spektar perspektiva i pristupa u vizualizaciji podataka o Nobelovim nagradama te mogu pružiti korisne uvide u statistiku i trendove povezane s Nobelovom nagradom. Navedeni izvori koriste tipove grafova i vizualizacije slične predloženima u prethodnom poglavlju.
\subsection{Boje i podaci}
Prilikom vizualizacije koristit će se jedna odabrana paleta boja, kako bi sve vizualizacije bile ujednačene I kako ne bi bilo previše boja, što bi uzrokovalo dodatne nejasnoće (promatrači će bojama pridavati određeno značenje I kontekst, a previše boja bi stvorilo previše nejasnoća I dodanog šuma u informaciji koju želimo prenijeti). Konkretno, s obzirom da je riječ o prepoznatljivom brendu, koristi će se paleta boja koja je tipična za sve vizuale koji sadrže nobelovu nagradu.
\newpage
\section{Izrada vizualizacije podataka}
Funkcionalnost se odnosi na mogućnosti ili akcije koje mogu biti izvršene pomoću softverske biblioteke ili okvira kao što je D3.js. Opisuje se što je biblioteka sposobna napraviti ili zadaci koje mogu biti obavljeni. U kontekstu D3.js, funkcionalnost može uključivati učitavanje podataka, manipulaciju podacima, stvaranje vizualizacija, obradu događaja i druge zadatke koji mogu biti postignuti korištenjem mogućnosti i metoda D3.js. S druge strane, ponašanje (behavior) u D3.js odnosi se na konkretne akcije ili odgovore koji su iskazani od strane stvorenih vizualnih elemenata ili komponenti pomoću D3.js-a. Opisuje se kako se elementi ili komponente ponašaju ili interaktiraju s korisničkim unosima, događajima ili promjenama podataka. Ponašanje u D3.js odnosi se na to kako vizualizacije reagiraju na korisničke interakcije kao što su događaji mouseover, klikovi, prijelazi ili animacije. U nastavku su izlistane osnovne i napredne funckionalnosti i ponašanja koje rješenje nudi, kao i implementacija osnovnih funkcionalnosti koje su potrebne za prototip vizualizacije.
\subsection{Jednostavne/složene funkcionalnosti i ponašanja}
Kroz praćenje kolegija, kao i kroz konzultacije s OpenAI modelom \cite{gpt3.5}, sastavljen je popis osnovnih i naprednih funkcionalnosti i ponašanja implementiranih u ovom radu:
\begin{enumerate}[label=\arabic*.]
  \item \textbf{Jednostavne (osnovne) funkcionalnosti}
  \begin{itemize}
    \item Definiraju se širina i visina kontejnera karte.
    \item Definira se putanja do GeoJSON datoteke.
    \item GeoJSON datoteka se učitava.
    \item CSV podaci se učitavaju i parsiraju.
    \item Postavlja se kontejner karte.
    \item Definira se projekcija i generator putanje.
    \item Stvara se SVG element za kartu.
    \item Stvara se element za prikazivanje tooltipa.
    \item Definira se trenutna kategorija.
    \item Ažuriraju se boje karte na temelju podataka.
    \item Inicijalno ažuriranje boja karte se poziva.
    \item Rukuje se događajem promjene odabira kategorije.
    \item Rukuje se događajem zumiranja.
  \end{itemize}

  \item \textbf{Kompleksne funkcionalnosti}
  \begin{itemize}
    \item Podaci se filtriraju na temelju odabrane kategorije.
    \item Podaci se grupiraju po zemlji i računaju se statistike.
    \item Stvara se skala boja za kartu.
    \item Crtaju se karte uz prijelaze.
    \item Putanja se ažurira uz prijelaz.
    \item Boja putanje se ažurira uz prijelaz.
    \item Rukuje se događajem klika za generiranje kružnog dijagrama za odabranu zemlju.
    \item Generira se kružni dijagram za odabranu zemlju.
    \item Briše se postojeći kružni dijagram.
    \item Podaci se filtriraju za odabranu zemlju i kategoriju.
    \item Podaci se grupiraju po spolu i računaju se statistike.
    \item Definiraju se dimenzije kružnog dijagrama.
    \item Stvara se SVG element za kružni dijagram.
    \item Definira se skala boja za kružni dijagram.
    \item Generiraju se podaci za kružni dijagram.
    \item Definira se raspored kružnog dijagrama.
    \item Generiraju se lukovi kružnog dijagrama.
    \item Crtaju se dijelovi kružnog dijagrama uz prijelaze.
    \item Dodaju se oznake na dijelove kružnog dijagrama.
    \item Dodaje se naslov kružnom dijagramu.
  \end{itemize}

  \item \textbf{Jednostavna (osnovna) ponašanja}
  \begin{itemize}
    \item Prikazuje se tooltip pri pomicanju miša iznad putanje.
    \item Tooltip se skriva kada miš napusti putanju.
  \end{itemize}

  \item \textbf{Kompleksna ponašanja}
  \begin{itemize}
    \item Lukovi kružnog dijagrama se interpoliraju radi glatkog prijelaza.
  \end{itemize}
\end{enumerate}
\subsection{Implementacija programske podrške}
\lstinputlisting[caption={JavaScript code snippet}, label={lst:javascript_code}]{script.js}
\newpage
\newpage
\printbibliography
\addcontentsline{toc}{section}{Literatura}
\end{document}